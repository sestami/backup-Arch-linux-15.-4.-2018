\documentclass[11pt,a4paper]{article}
\documentclass[11pt,a4paper]{article}
\usepackage[utf8]{inputenc}
\usepackage[czech]{babel}
\usepackage{amsmath}
\usepackage{amsfonts}
\usepackage{booktabs}
\usepackage{longtable}
\usepackage{amssymb}
\usepackage{graphicx}
\usepackage{float}
\usepackage{subcaption}
\usepackage[left=2cm,right=2cm,top=4cm,bottom=4cm]{geometry}
\author{Michal Šesták}
\title{<++>}
\usepackage[unicode, pdftex, pdfauthor={Michal Šesták}, pdftitle={<++>}, pdfsubject={PDZ},colorlinks=true, linkcolor=red,
urlcolor=blue, citecolor=blue]{hyperref}
\LTcapwidth=\textwidth
\renewcommand{\arraystretch}{1.0}
\begin{document}

\renewcommand{\figurename}{Obr.}
\renewcommand{\tablename}{Tab.}

\LARGE
\begin{center}
\textbf{Praktikum z detekce a dozimetrie ioniz. záření\\ FJFI ČVUT v Praze}
\end{center}\normalsize

\shorthandoff{-}
\begin{center}
\begin{tabular}{l l}
\multicolumn{2}{l}{Název úlohy: <++>}\\
\hline
Datum: <++> 2017 & Vypracoval: Michal Šesták \\
Číslo úlohy: <++> & Skupina: 1 \\
Čas praktik: Středa, 9.30 & Klasifikace:\\
\multicolumn{2}{l}{Spoluměřící: Kristýna Gincelová, Mahulena Kuklová, Bára Vendlová}\\
\end{tabular}
\end{center}
\shorthandon{-}

\section{Pracovní úkoly}
\begin{enumerate}
\item <++> 
\item <++> 
\item <++> 
\item <++> 
\end{enumerate}
\section{Použité přístroje a pomůcky}
<++>
\section{Teoretický úvod}
<++>
\section{Postup měření}
<++>
\section{Výsledky měření}
<++>
\section{Diskuze}
<++>
\section{Závěr}
<++>
\begin{thebibliography}{Mm99}
\bibitem{behounek} FJFI: Úloha PDDZ č. <++>: <+nazev+>. <+pocetStran+> s. Citováno <++> 2017, dostupné z: \url{https://behounek.fjfi.cvut.cz/}
\end{thebibliography}

%\pagestyle{empty}
%\section{Přílohy}
%\appendix
%\section{Zpracování chyb měření}
%Při statistickém zpracování naměřených dat používáme vztahy pro aritmetický průměr a střední kvadratickou chybu aritmetického průměru
%\begin{align}
%\overline{z}&=\frac{1}{n}\sum_{i=1}^n{z_i},\label{eq:PRUMER}\\
%\sigma_{\overline{z}}&=\sqrt{\frac{\sum_{i=1}^n{(\overline{z}-z_i})^2}{n(n-1)}},
%\label{eq:CHYBAprima}
%\end{align}
%kde $n$ je počet naměřených hodnot veličiny $z$.


%Máme-li zjistit nepřesnost veličiny $f=f(x_1,x_2,\dots)$ závislé na veličinách $x_1,x_2,\dots$ ze známosti hodnot $x_1,x_2,\dots$ s jejich chybami $\sigma_{x_1}, \sigma_{x_2},\dots$, pak využijeme vzorec
%\begin{equation}
%\sigma_f=\sqrt{\left( \frac{\partial f}{\partial x_1} \right)^2\sigma_{x_1}^2+\left( \frac{\partial f}{\partial x_2} \right)^2\sigma_{x_2}^2+\dots}.
%\label{eq:CHYBAneprima}
%\end{equation}
\end{document}


